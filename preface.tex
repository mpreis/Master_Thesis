\hypersetup{pageanchor = false}%

%%%%%%%%%%%%%%%%%%%%%%%%%%%%%%%%%%%%%%%%
% titlepage / Titelseite
\maketitle%

%%%%%%%%%%%%%%%%%%%%%%%%%%%%%%%%%%%%%%%%
% statement of authentication / Verfassungserklaerung
\authenticationstatement%

%%%%%%%%%%%%%%%%%%%%%%%%%%%%%%%%%%%%%%%%
% acknowledgments / Danksagung
\acknowledgments{
Es ist an der Zeit einigen Begleitern auf meinem Weg dankzusagen.
\\~\\
% Christoph Kirsch
Ein großes Dankeschön gebührt meinem Betreuer \emph{Prof. Christoph Kirsch} für die Betreuung durch mein Beachelor- und Masterstudium. Du hast meine Entwicklung voran getrieben und mich stets zu Höchstleistungen motiviert, du hast mir Türen geöffnet von denen ich nicht zu träumen gewagt hätte, für diese wertvollen Erfahrungen möchte ich dir danken.
\\~\\
% Mama & Papa
Am Ende meines Studiums angekommen möchte ich mich bei meiner \emph{Familie} bedanken. Insbesondere gilt der Danke meinen Eltern, Günther und Helga, für ihr Vertrauen und ihre Unterstützung. Danke, dass ihr mir dieses Studium ermöglicht habt.
\\~\\
% Alexander Miller
Dankeschön auch an \emph{Alexander Miller} für die wertvollen Diskussionen welche dieses Projekt vorangetrieben haben.
\\~\\
% Thomas Hütter
Abschließend möchte ich mich noch bei \emph{Thomas Hütter} bedanken, für die unzähligen Stunden die wir mit Projekten verbracht haben.
\\~\\
\textbf{Danke.}
}

%%%%%%%%%%%%%%%%%%%%%%%%%%%%%%%%%%%%%%%%
% abstract / Kurzfassung
\abstract{%
% motivation
A known problem in computer systems is the latency of accessing data stored in the main memory.
Computer architects analyzed real programs to determine a sophisticate solution to reduce the memory access time and they introduced the concept of a \emph{cache}.
Nowadays the perspective changed, scientists analyze the behavior of caches to push a programs performance to its maximum.
% problem statement
The question is, can we find metrics that characterize the performance of a program for a given cache.
% methods, approach
We analyze the sequence of \emph{load} and \emph{store} instructions, the memory access trace, of well known benchmark suites, SPEC 2006 and Octane, about there characteristics.
In this work we use four metrics: (1) the number of accesses on an address (\emph{Accesses}), (2) the distance between two accesses on the same address (\emph{Access Distance}), (3) the timespan an address is in use (\emph{Liveness Interval Length}), and (4) the number of overlapping liveness intervals at a certain point of the execution (\emph{Overlapping Liveness}).
We transform the observed traces into semantically equivalent traces and executed the on different cache simulations to find potential performance improvements.
% results
Our experiments show that traces with extremely short liveness intervals for most addresses and few accesses on those provide the highest potential for improving the performance significantly.
Traces with short access distances and short liveness intervals for most addresses provide the highest potential for reducing the total number of addresses used.
% conclusion
In this works we show that the trace of a program characterizes its performance. Further, our analysis show that the trace of a program reveals a programs potential for improvement in terms of memory access performance and memory usage.
}

%%%%%%%%%%%%%%%%%%%%%%%%%%%%%%%%%%%%%%%%
% table of contents / Inhaltsverzeichnis
\setcounter{tocdepth}{5}
\tableofcontents%

\hypersetup{pageanchor = true}%
