\hypersetup{pageanchor = false}%

%%%%%%%%%%%%%%%%%%%%%%%%%%%%%%%%%%%%%%%%
% titlepage / Titelseite
\maketitle%

%%%%%%%%%%%%%%%%%%%%%%%%%%%%%%%%%%%%%%%%
% statement of authentication / Verfassungserklaerung
\authenticationstatement%

%%%%%%%%%%%%%%%%%%%%%%%%%%%%%%%%%%%%%%%%
% acknowledgments / Danksagung
\acknowledgments{
Es ist an der Zeit einigen Begleitern auf meinem Weg dankzusagen.
\\~\\
% Christoph Kirsch
Ein großes Dankeschön gebührt meinem Betreuer \emph{Prof. Christoph Kirsch} für die Betreuung durch mein Beachelor- und Masterstudium. Du hast meine Entwicklung voran getrieben und mich stets zu Höchstleistungen motiviert, du hast mir Türen geöffnet von denen ich nicht zu träumen gewagt hätte, für diese wertvollen Erfahrungen möchte ich dir danken.
\\~\\
% Mama & Papa
Am Ende meines Studiums angekommen möchte ich mich bei meiner \emph{Familie} bedanken. Insbesondere gilt der Danke meinen Eltern, Günther und Helga, für ihr Vertrauen und ihre Unterstützung. Danke, dass ihr mir dieses Studium ermöglicht habt.
\\~\\
% Alexander Miller
Dankeschön auch an \emph{Alexander Miller} für die wertvollen Diskussionen welche dieses Projekt vorangetrieben haben.
\\~\\
% Thomas Hütter
Abschließend möchte ich mich noch bei \emph{Thomas Hütter} bedanken, für die unzähligen Stunden die wir mit Projekten verbracht haben.
\\~\\
\textbf{Danke.}
}

%%%%%%%%%%%%%%%%%%%%%%%%%%%%%%%%%%%%%%%%
% abstract / Kurzfassung
\abstract{%
% motivation
A known problem in computer systems is the latency of accessing data stored in the main memory.
% problem statement
The question is, can we find metrics that characterize the performance of a program for a given cache.
% methods & approach
We analyze the sequence of \emph{load} and \emph{store} instructions, the \emph{memory access trace}, of well known benchmark suites, SPEC 2006 and V8, about there characteristics.
For our analysis about the potential for performance improvement we modify the addresses used by a memory access trace.
% results & conclusion
Our analysis illustrate that the memory access trace of a program reveals a programs potential for improvement in terms of memory access performance and memory usage.
Nonetheless, the chose metrics illustrate tendencies for improvement rather than unique characteristics.
}

%%%%%%%%%%%%%%%%%%%%%%%%%%%%%%%%%%%%%%%%
% table of contents / Inhaltsverzeichnis
\setcounter{tocdepth}{5}
\tableofcontents%

\hypersetup{pageanchor = true}%
