\hypersetup{pageanchor = false}%

%%%%%%%%%%%%%%%%%%%%%%%%%%%%%%%%%%%%%%%%
% titlepage / Titelseite
\maketitle%

%%%%%%%%%%%%%%%%%%%%%%%%%%%%%%%%%%%%%%%%
% statement of authentication / Verfassungserklaerung
\authenticationstatement%

%%%%%%%%%%%%%%%%%%%%%%%%%%%%%%%%%%%%%%%%
% acknowledgments / Danksagung
\acknowledgments{
Es ist an der Zeit einigen Begleitern auf meinem Weg dankzusagen.
\\~\\
% Christoph Kirsch
Ein großes Dankeschön gebührt meinem Betreuer \emph{Prof. Christoph Kirsch} für die Betreuung durch mein Bachelor- und Masterstudium. Du hast meine Entwicklung voran getrieben und mich stets zu Höchstleistungen motiviert; du hast mir Türen geöffnet von denen ich nicht zu träumen gewagt hätte; für diese wertvollen Erfahrungen möchte ich dir danken.
\\~\\
% Mama & Papa
Am Ende meines Studiums angekommen, möchte ich mich bei meiner \emph{Familie} bedanken. Insbesondere gilt der Dank meinen Eltern, Günther und Helga, für ihr Vertrauen und ihre Unterstützung. Danke, dass ihr mir dieses Studium ermöglicht habt.
\\~\\
% Alexander Miller
Dankeschön auch an \emph{Alexander Miller} für die wertvollen Diskussionen, welche dieses Projekt vorangetrieben haben.
\\~\\
% Thomas Hütter
Abschließend möchte ich mich noch bei \emph{Thomas Hütter} bedanken; für die unzähligen Stunden, die wir mit Projekten verbracht haben.
\\~\\
\textbf{Danke.}
}

%%%%%%%%%%%%%%%%%%%%%%%%%%%%%%%%%%%%%%%%
% abstract / Kurzfassung
\abstract{%
% motivation
The latency of accessing data stored in the main memory is a known problem in computer systems.
% problem statement
We are interested in finding metrics that characterize the performance of a program for a given cache.
% methods & approach
We analyze the characteristic of \emph{load} and \emph{store} instructions, called the \emph{memory access trace}, of two well known benchmark suites, namely SPEC 2006 and V8.
For our analysis about the potential performance improvement in terms of memory access performance and memory usage we modify the addresses used by a memory access trace.
% results & conclusion
Our analysis illustrates that for some benchmarks we are able to improve the memory usage and the memory access performance by a factor of at least two.
Nonetheless, the chosen metrics illustrate tendencies for improvement rather than unique characteristics.
}

%%%%%%%%%%%%%%%%%%%%%%%%%%%%%%%%%%%%%%%%
% table of contents / Inhaltsverzeichnis
\setcounter{tocdepth}{5}
\tableofcontents%

\hypersetup{pageanchor = true}%
